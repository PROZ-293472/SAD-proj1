% Options for packages loaded elsewhere
\PassOptionsToPackage{unicode}{hyperref}
\PassOptionsToPackage{hyphens}{url}
%
\documentclass[
]{article}
\usepackage{amsmath,amssymb}
\usepackage{lmodern}
\usepackage{iftex}
\ifPDFTeX
  \usepackage[T1]{fontenc}
  \usepackage[utf8]{inputenc}
  \usepackage{textcomp} % provide euro and other symbols
\else % if luatex or xetex
  \usepackage{unicode-math}
  \defaultfontfeatures{Scale=MatchLowercase}
  \defaultfontfeatures[\rmfamily]{Ligatures=TeX,Scale=1}
\fi
% Use upquote if available, for straight quotes in verbatim environments
\IfFileExists{upquote.sty}{\usepackage{upquote}}{}
\IfFileExists{microtype.sty}{% use microtype if available
  \usepackage[]{microtype}
  \UseMicrotypeSet[protrusion]{basicmath} % disable protrusion for tt fonts
}{}
\makeatletter
\@ifundefined{KOMAClassName}{% if non-KOMA class
  \IfFileExists{parskip.sty}{%
    \usepackage{parskip}
  }{% else
    \setlength{\parindent}{0pt}
    \setlength{\parskip}{6pt plus 2pt minus 1pt}}
}{% if KOMA class
  \KOMAoptions{parskip=half}}
\makeatother
\usepackage{xcolor}
\IfFileExists{xurl.sty}{\usepackage{xurl}}{} % add URL line breaks if available
\IfFileExists{bookmark.sty}{\usepackage{bookmark}}{\usepackage{hyperref}}
\hypersetup{
  pdftitle={SAD Projekt 1 - Sprawozdanie},
  pdfauthor={Michal Szpunar, Daniel Adamkowski},
  hidelinks,
  pdfcreator={LaTeX via pandoc}}
\urlstyle{same} % disable monospaced font for URLs
\usepackage[margin=1in]{geometry}
\usepackage{color}
\usepackage{fancyvrb}
\newcommand{\VerbBar}{|}
\newcommand{\VERB}{\Verb[commandchars=\\\{\}]}
\DefineVerbatimEnvironment{Highlighting}{Verbatim}{commandchars=\\\{\}}
% Add ',fontsize=\small' for more characters per line
\usepackage{framed}
\definecolor{shadecolor}{RGB}{248,248,248}
\newenvironment{Shaded}{\begin{snugshade}}{\end{snugshade}}
\newcommand{\AlertTok}[1]{\textcolor[rgb]{0.94,0.16,0.16}{#1}}
\newcommand{\AnnotationTok}[1]{\textcolor[rgb]{0.56,0.35,0.01}{\textbf{\textit{#1}}}}
\newcommand{\AttributeTok}[1]{\textcolor[rgb]{0.77,0.63,0.00}{#1}}
\newcommand{\BaseNTok}[1]{\textcolor[rgb]{0.00,0.00,0.81}{#1}}
\newcommand{\BuiltInTok}[1]{#1}
\newcommand{\CharTok}[1]{\textcolor[rgb]{0.31,0.60,0.02}{#1}}
\newcommand{\CommentTok}[1]{\textcolor[rgb]{0.56,0.35,0.01}{\textit{#1}}}
\newcommand{\CommentVarTok}[1]{\textcolor[rgb]{0.56,0.35,0.01}{\textbf{\textit{#1}}}}
\newcommand{\ConstantTok}[1]{\textcolor[rgb]{0.00,0.00,0.00}{#1}}
\newcommand{\ControlFlowTok}[1]{\textcolor[rgb]{0.13,0.29,0.53}{\textbf{#1}}}
\newcommand{\DataTypeTok}[1]{\textcolor[rgb]{0.13,0.29,0.53}{#1}}
\newcommand{\DecValTok}[1]{\textcolor[rgb]{0.00,0.00,0.81}{#1}}
\newcommand{\DocumentationTok}[1]{\textcolor[rgb]{0.56,0.35,0.01}{\textbf{\textit{#1}}}}
\newcommand{\ErrorTok}[1]{\textcolor[rgb]{0.64,0.00,0.00}{\textbf{#1}}}
\newcommand{\ExtensionTok}[1]{#1}
\newcommand{\FloatTok}[1]{\textcolor[rgb]{0.00,0.00,0.81}{#1}}
\newcommand{\FunctionTok}[1]{\textcolor[rgb]{0.00,0.00,0.00}{#1}}
\newcommand{\ImportTok}[1]{#1}
\newcommand{\InformationTok}[1]{\textcolor[rgb]{0.56,0.35,0.01}{\textbf{\textit{#1}}}}
\newcommand{\KeywordTok}[1]{\textcolor[rgb]{0.13,0.29,0.53}{\textbf{#1}}}
\newcommand{\NormalTok}[1]{#1}
\newcommand{\OperatorTok}[1]{\textcolor[rgb]{0.81,0.36,0.00}{\textbf{#1}}}
\newcommand{\OtherTok}[1]{\textcolor[rgb]{0.56,0.35,0.01}{#1}}
\newcommand{\PreprocessorTok}[1]{\textcolor[rgb]{0.56,0.35,0.01}{\textit{#1}}}
\newcommand{\RegionMarkerTok}[1]{#1}
\newcommand{\SpecialCharTok}[1]{\textcolor[rgb]{0.00,0.00,0.00}{#1}}
\newcommand{\SpecialStringTok}[1]{\textcolor[rgb]{0.31,0.60,0.02}{#1}}
\newcommand{\StringTok}[1]{\textcolor[rgb]{0.31,0.60,0.02}{#1}}
\newcommand{\VariableTok}[1]{\textcolor[rgb]{0.00,0.00,0.00}{#1}}
\newcommand{\VerbatimStringTok}[1]{\textcolor[rgb]{0.31,0.60,0.02}{#1}}
\newcommand{\WarningTok}[1]{\textcolor[rgb]{0.56,0.35,0.01}{\textbf{\textit{#1}}}}
\usepackage{graphicx}
\makeatletter
\def\maxwidth{\ifdim\Gin@nat@width>\linewidth\linewidth\else\Gin@nat@width\fi}
\def\maxheight{\ifdim\Gin@nat@height>\textheight\textheight\else\Gin@nat@height\fi}
\makeatother
% Scale images if necessary, so that they will not overflow the page
% margins by default, and it is still possible to overwrite the defaults
% using explicit options in \includegraphics[width, height, ...]{}
\setkeys{Gin}{width=\maxwidth,height=\maxheight,keepaspectratio}
% Set default figure placement to htbp
\makeatletter
\def\fps@figure{htbp}
\makeatother
\setlength{\emergencystretch}{3em} % prevent overfull lines
\providecommand{\tightlist}{%
  \setlength{\itemsep}{0pt}\setlength{\parskip}{0pt}}
\setcounter{secnumdepth}{-\maxdimen} % remove section numbering
\ifLuaTeX
  \usepackage{selnolig}  % disable illegal ligatures
\fi

\title{SAD Projekt 1 - Sprawozdanie}
\author{Michal Szpunar, Daniel Adamkowski}
\date{30 11 2021}

\begin{document}
\maketitle

\hypertarget{zadanie-1}{%
\subsection{Zadanie 1}\label{zadanie-1}}

Analizę dokonano dla zbioru danych z lipca 2021 roku. Poniżej
przedstawiono procedurę wczytania oraz oznaczenia danych:

\begin{Shaded}
\begin{Highlighting}[]
\NormalTok{schema }\OtherTok{\textless{}{-}} \FunctionTok{c}\NormalTok{(}
  \StringTok{\textquotesingle{}station\_id\textquotesingle{}}\NormalTok{,}
  \StringTok{\textquotesingle{}station\_name\textquotesingle{}}\NormalTok{,}
  \StringTok{\textquotesingle{}year\textquotesingle{}}\NormalTok{,}
  \StringTok{\textquotesingle{}month\textquotesingle{}}\NormalTok{,}
  \StringTok{\textquotesingle{}day\textquotesingle{}}\NormalTok{,}
  \StringTok{\textquotesingle{}max\_daily\_temp\textquotesingle{}}\NormalTok{,}
  \StringTok{\textquotesingle{}meas\_status\_TMAX\textquotesingle{}}\NormalTok{,}
  \StringTok{\textquotesingle{}min\_daily\_temp\textquotesingle{}}\NormalTok{,}
  \StringTok{\textquotesingle{}meas\_status\_TMIN\textquotesingle{}}\NormalTok{,}
  \StringTok{\textquotesingle{}mean\_daily\_temp\textquotesingle{}}\NormalTok{,}
  \StringTok{\textquotesingle{}meas\_status\_STD\textquotesingle{}}\NormalTok{,}
  \StringTok{\textquotesingle{}min\_temp\_near\_ground\textquotesingle{}}\NormalTok{,}
  \StringTok{\textquotesingle{}meas\_status\_TMNG\textquotesingle{}}\NormalTok{,}
  \StringTok{\textquotesingle{}daily\_rainfall\textquotesingle{}}\NormalTok{,}
  \StringTok{\textquotesingle{}meas\_status\_SMDB\textquotesingle{}}\NormalTok{,}
  \StringTok{\textquotesingle{}rainfall\_type\textquotesingle{}}\NormalTok{,}
  \StringTok{\textquotesingle{}snow\_cover\_height\textquotesingle{}}\NormalTok{,}
  \StringTok{\textquotesingle{}meas\_status\_PKSN\textquotesingle{}}
  
\NormalTok{)}

\NormalTok{df }\OtherTok{\textless{}{-}} \FunctionTok{read.csv}\NormalTok{(}\StringTok{\textquotesingle{}./data/2021\_07\_k/k\_d\_07\_2021.csv\textquotesingle{}}\NormalTok{, }\AttributeTok{header=}\ConstantTok{FALSE}\NormalTok{)}
\FunctionTok{colnames}\NormalTok{(df) }\OtherTok{\textless{}{-}}\NormalTok{ schema}
\end{Highlighting}
\end{Shaded}

Następnie stworzono kolumny zawierające różnice pomiędzy maksymalnymi
oraz średnimi temperaturami z następujących po sobie dni

\begin{Shaded}
\begin{Highlighting}[]
\NormalTok{df }\OtherTok{\textless{}{-}}\NormalTok{ df }\SpecialCharTok{\%\textgreater{}\%} \FunctionTok{group\_by}\NormalTok{(station\_name) }\SpecialCharTok{\%\textgreater{}\%} \FunctionTok{mutate}\NormalTok{(}\AttributeTok{max\_temp\_diff =}\NormalTok{ max\_daily\_temp }\SpecialCharTok{{-}} \FunctionTok{lag}\NormalTok{(max\_daily\_temp))}
\NormalTok{df }\OtherTok{\textless{}{-}}\NormalTok{ df }\SpecialCharTok{\%\textgreater{}\%} \FunctionTok{group\_by}\NormalTok{(station\_name) }\SpecialCharTok{\%\textgreater{}\%} \FunctionTok{mutate}\NormalTok{(}\AttributeTok{mean\_temp\_diff =}\NormalTok{ mean\_daily\_temp }\SpecialCharTok{{-}} \FunctionTok{lag}\NormalTok{(mean\_daily\_temp))}
\end{Highlighting}
\end{Shaded}

\hypertarget{including-plots}{%
\subsection{Including Plots}\label{including-plots}}

You can also embed plots, for example:

\includegraphics{sprawko_files/figure-latex/pressure-1.pdf}

Note that the \texttt{echo\ =\ FALSE} parameter was added to the code
chunk to prevent printing of the R code that generated the plot.

\end{document}
